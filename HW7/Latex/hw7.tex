\documentclass[12pt,fullpage,letterpaper]{article}

\newenvironment{proof}{\noindent{\bf Proof:}}{\qed\bigskip}

\newtheorem{theorem}{Theorem}
\newtheorem{corollary}{Corollary}
\newtheorem{lemma}{Lemma} 
\newtheorem{claim}{Claim}
\newtheorem{fact}{Fact}
\newtheorem{definition}{Definition}
\newtheorem{assumption}{Assumption}
\newtheorem{observation}{Observation}
\newtheorem{example}{Example}
\newcommand{\qed}{\rule{7pt}{7pt}}

\newcommand{\assignment}[4]{
\thispagestyle{plain} 
\newpage
\setcounter{page}{1}
\noindent
\begin{center}
\framebox{ \vbox{ \hbox to 6.28in
{\bf CS446: Machine Learning \hfill #1}
\vspace{4mm}
\hbox to 6.28in
{\hspace{2.5in}\large\mbox{Problem Set #2}}
\vspace{4mm}
\hbox to 6.28in
{{\it Handed Out: #3 \hfill Due: #4}}
}}
\end{center}
}


\newcommand{\handout}[3]{
\thispagestyle{plain} 
\newpage
\setcounter{page}{1}
\noindent
\begin{center}
\framebox{ \vbox{ \hbox to 6.28in
{\bf CS446: Machine Learning \hfill #1}
\vspace{4mm}
\hbox to 6.28in
{\hspace{2.5in}\large\mbox{#2}}
\vspace{4mm}
\hbox to 6.28in
{{\it Handed Out: #3 \hfill Name (NetID): \rule[-2pt]{4cm}{0.1pt} }}
}}
\end{center}
}


\newcommand{\assgsoln}[4]{
\thispagestyle{plain} 
\newpage
\setcounter{page}{1}
\noindent
\begin{center}
\framebox{ \vbox{ \hbox to 6.28in
{\bf CS446: Machine Learning \hfill #1}
\vspace{4mm}
\hbox to 6.28in
{\hspace{2.5in}\large\mbox{Problem Set #2 Solutions}}
\vspace{4mm}
\hbox to 6.28in
{{\it Handed Out: #3 \hfill Handed In: #4}}
}}
\end{center}
}


\newcommand{\solution}[4]{
\thispagestyle{plain} 
\newpage
\setcounter{page}{1}
\noindent
\begin{center}
\framebox{ \vbox{ \hbox to 6.28in
{\bf CS446: Machine Learning \hfill #4}
\vspace{4mm}
\hbox to 6.28in
{\hspace{2.5in}\large\mbox{Problem Set #3}}
\vspace{4mm}
\hbox to 6.28in
{#1 \hfill {\it Handed In: #2}}
}}
\end{center}
\markright{#1}
}


\newenvironment{algorithm}
{\begin{center}
\begin{tabular}{|l|}
\hline
\begin{minipage}{1in}
\begin{tabbing}
\quad\=\qquad\=\qquad\=\qquad\=\qquad\=\qquad\=\qquad\=\kill}
{\end{tabbing}
\end{minipage} \\
\hline
\end{tabular}
\end{center}}

\def\Comment#1{\textsf{\textsl{$\langle\!\langle$#1\/$\rangle\!\rangle$}}}


\usepackage{amsmath,url,graphicx,amssymb}
\usepackage{xcolor}
\sloppy

\oddsidemargin 0in
\evensidemargin 0in
\textwidth 6.5in
\topmargin -0.5in
\textheight 9.0in

\newcommand{\tight}[1]{\!#1\!}
\newcommand{\loose}[1]{\;#1\;}

\begin{document}

\assignment{SPRING 2017}{7}{April $20^{th}$, $2017$}{May $1^{st}$, $2017$}

\begin{footnotesize}
\begin{itemize}
\item Feel free to talk to other members of the class in doing the homework.
I am more concerned that you learn how to solve the problem than that you
demonstrate that you solved it entirely on your own.  You should, however,
write down your solution yourself.  Please try to keep the solution brief
and clear.

\item Please use Piazza first if you have questions about the homework.
Also feel free to send us e-mails and come to office hours.

\item Please, no handwritten solutions.  You will submit your solution
manuscript as a single pdf file.

\item The homework is due at \textbf{11:59 PM} on the due date. We will be
using Compass for collecting the homework assignments. Please submit an
electronic copy via Compass2g (\texttt{http://compass2g.illinois.edu}).
Please do NOT hand in a hard copy of your write-up.  Contact the TAs if you
are face technical difficulties in submitting the assignment.

\item \textcolor{red}{You cannot use the late submission credit hours for
this problem set.}

\item No code is needed for any of these problems. You can do the
calculations however you please. You need to turn in only the report. Please
name your report as \texttt{$\langle$NetID$\rangle$-hw7.pdf}.
\end{itemize}
\end{footnotesize}

\begin{enumerate}
\item {\bf [EM Algorithm - 70 points]}

Assume we have a set $D$ of $m$ data points, where for each data point $x$ from
$D$, $x\in\{0,1\}^{n+1}$.  Denote the $i$-th bit of the $j$-th example as
$x^{(j)}_{i}$.  Thus, the index $i$ ranges from $0 \ldots n$, and the index
$j$ ranges from $1 \ldots m$.

Assume these data points were generated according to the following
distribution:

Postulate a hidden random variable $Z$ with values $z = 1, 2$, where the
probability of $z=1$ is $\alpha$ and the probability of $z=2$ is $1-\alpha$,
where $0<\alpha<1$.

For a specific example $x^{(j)}$, a random value of $Z$ is chosen, but its
true value $z$ is hidden.  Note that each example $x^{(j)}$ has a fixed
underlying $z$.  If $z=1$, $x^{(j)}_i$ is set to 1 with probability $p_i$.  If
$z=2$, the bit is set to 1 with probability $q_i$.  Thus, there are $2n+3$
unknown parameters.  You will use EM to develop an algorithm to estimate these
unknown parameters.

\begin{enumerate}
\item{\bf [10 points]}
Express $\Pr(x^{(j)})$ first in terms of conditional probabilities and then in
terms of the unknown parameters $\alpha$, $p_i$, and $q_i$.

\item{\bf [10 points]}
Let $f^{(j)}_z = \Pr(Z=z\mid x^{(j)})$, i.e.\ the probability that the data
point $x^{(j)}$ has $z$ as the value of its hidden variable $Z$. Express
$f^{(j)}_1$ and $f^{(j)}_2$ in terms of the unknown parameters.

\item{\bf [10 points]}
Derive an expression for the expected log likelihood ($E[LL]$) of the entire
data set $D$ and its associated $z$ settings given new parameter estimates
$\tilde{\alpha}, \tilde{p_i}, \tilde{q_i}$.

\item{\bf [10 points]}
Maximize the log likelihood ($LL$) and determine the update rules for the
parameters according to the EM algorithm.

\item{\bf [10 points]}
Examine the update rules explain them in English.  Describe in
pseudocode how you would run the algorithm: initialization, iteration,
termination.  What equations would you use at which steps in the algorithm?

\item{\bf [10 points]}
Assume that your task is to predict the value of $x_0$ given an assignment to
the other $n$ variables and that you have the parameters of the model.  Show
how to use these parameters to predict $x_0$.  ({\it Hint:} Consider the ratio
between $P(X_0=0)$ and $P(X_0=1)$.)

\item{\bf [10 points]}
Show that the decision surface for this prediction is a linear function of
the $x_i$'s.
\end{enumerate}





    \item {\bf [Tree Dependent Distributions - 30 points]}

      % Assume an undirected tree $T$ obtained by the algorithm described in class for
      % learning tree dependent distributions.  We would like to show that the step of
      % directing the tree by choosing an arbitrary node is okay.

      %{\bf Note:} In this problem, we will be looking at tree dependent
      %distributions that will be covered in class soon. You may go through the
      %lecture notes or wait for it to be taught in class before you attempt this
      %problem. A brief introduction is given below.

      A tree dependent distribution is a probability distribution over $n$
      variables, $\{x_1,\ldots,x_n\}$ that can be represented as a tree built
      over $n$ nodes corresponding to the variables. If there is a directed edge
      from variable $x_i$ to variable $x_j$, then $x_i$ is said to be the parent
      of $x_j$. Each directed edge $\langle x_i, x_j\rangle$ has a weight that
      indicates the conditional probability $\Pr(x_j \loose{|} x_i)$. In addition,
      we also have probability $\Pr(x_r)$ associated with the root node $x_r$.
      While computing joint probabilities over tree-dependent distributions, we
      assume that a node is independent of all its non-descendants given its
      parent. For instance, in our example above, $x_j$ is independent of all its
      non-descendants given $x_i$.

      To learn a tree-dependent distribution, we need to learn three things: the
      structure of the tree, the conditional probabilities on the edges of the tree, and the
      probabilities on the nodes.  Assume that you have an algorithm to learn an
      {\em undirected} tree $T$ with all required probabilities. To clarify, for
      all {\em undirected} edges $\langle x_i, x_j\rangle$, we have learned both
      probabilities, $\Pr(x_i \loose{|} x_j)$ and $\Pr(x_j \loose{|} x_i)$.
      (There exists such an algorithm and we will be covering that in class.) The
      only aspect missing is the directionality of edges to convert this
      undirected tree to a directed one.

      However, it is okay to not learn the directionality of the edges explicitly. In
      this problem, you will show that choosing any arbitrary node as the root
      and directing all edges away from it is sufficient, and that two directed
      trees obtained this way from the same underlying undirected tree $T$ are
      equivalent.

      \begin{enumerate}
        \item {\bf [10 points]}
          State exactly what is meant by the statement: ``\emph{The two directed trees
          obtained from $T$ are equivalent.}''

        \item {\bf [20 points]}
          Show that no matter which node in $T$ is chosen as the root for the
          ``direction'' stage, the resulting directed trees are all equivalent (based
          on your definition above).

      \end{enumerate}


\end{enumerate}
\end{document}
